\nonstopmode{}
\documentclass[a4paper]{book}
\usepackage[times,inconsolata,hyper]{Rd}
\usepackage{makeidx}
\makeatletter\@ifl@t@r\fmtversion{2018/04/01}{}{\usepackage[utf8]{inputenc}}\makeatother
% \usepackage{graphicx} % @USE GRAPHICX@
\makeindex{}
\begin{document}
\chapter*{}
\begin{center}
{\textbf{\huge Package `FunJaccR'}}
\par\bigskip{\large \today}
\end{center}
\ifthenelse{\boolean{Rd@use@hyper}}{\hypersetup{pdftitle = {FunJaccR: Interpretation of Gene Lists Using Clustering of Functionally Enriched Terms}}}{}
\ifthenelse{\boolean{Rd@use@hyper}}{\hypersetup{pdfauthor = {Adam Reid}}}{}
\begin{description}
\raggedright{}
\item[Type]\AsIs{Package}
\item[Title]\AsIs{Interpretation of Gene Lists Using Clustering of Functionally
Enriched Terms}
\item[Version]\AsIs{0.1.0}
\item[Description]\AsIs{Given a gene list, funtionally enriched terms are determined using gProfiler.
MCL is used to cluster these terms based on the Jaccard similarity of the lists of genes associated with them.
This can be visualised in Cytoscape using the RCy3 package.}
\item[License]\AsIs{GPL (>=3)}
\item[Encoding]\AsIs{UTF-8}
\item[LazyData]\AsIs{true}
\item[RoxygenNote]\AsIs{7.3.2}
\item[Imports]\AsIs{MCL, RCy3, gprofiler2, gRbase, stringr}
\item[NeedsCompilation]\AsIs{no}
\item[Author]\AsIs{Adam Reid [aut, cre]}
\item[Maintainer]\AsIs{Adam Reid }\email{ajr236@cam.ac.uk}\AsIs{}
\item[Depends]\AsIs{R (>= 3.5.0)}
\end{description}
\Rdcontents{Contents}
\HeaderA{create\_cytoscape\_network}{Create a FunJacc network in Cytoscape}{create.Rul.cytoscape.Rul.network}
%
\begin{Description}
Draw the FunJacc results as a network in an open instance of Cytoscape running on the same machine
\end{Description}
%
\begin{Usage}
\begin{verbatim}
create_cytoscape_network(
  funjacc_results,
  title = "my first network",
  collection = "Funjacc Networks"
)
\end{verbatim}
\end{Usage}
%
\begin{Arguments}
\begin{ldescription}
\item[\code{funjacc\_results}] Funjacc results

\item[\code{title}] Network title

\item[\code{collection}] Network collection
\end{ldescription}
\end{Arguments}
%
\begin{Value}
Sets up a network in your Cytoscape application
\end{Value}
%
\begin{Examples}
\begin{ExampleCode}
create_cytoscape_network(funjacc_res, title="Test network")
\end{ExampleCode}
\end{Examples}
\HeaderA{funjacc}{Run FunJacc}{funjacc}
%
\begin{Description}
Determine enriched functional term clusters with FunJacc
\end{Description}
%
\begin{Usage}
\begin{verbatim}
funjacc(
  gene_list,
  p_cut = 0.01,
  jaccard_cut = 0.5,
  data_types = c("GO:BP"),
  inflation = 2,
  species = "hsapiens",
  default_node_label_size = 10,
  cluster_label_size = 20
)
\end{verbatim}
\end{Usage}
%
\begin{Arguments}
\begin{ldescription}
\item[\code{gene\_list}] List of genes in which to look for enriched terms

\item[\code{p\_cut}] P-value cutoff for gProfiler results

\item[\code{jaccard\_cut}] Jaccard index cutoff for identifying high scoring links between terms

\item[\code{data\_types}] List of functional terms to include e.g.c('GO:BP', 'GO:MF', 'GO:CC', 'KEGG', 'REAC', 'TF', 'MIRNA', 'CORUM', 'HP', 'HPA', 'WP'). Supplying 'all' will include all these terms

\item[\code{inflation}] Inflation parameter for MCL clustering \textasciitilde{}0.5-2

\item[\code{species}] Organism to use for gProfiler to interpret gene list e.g. 'hsapiens', 'mmusculus'

\item[\code{default\_node\_label\_size}] Node size to use for Cytoscape plotting

\item[\code{cluster\_label\_size}] Node size for cluster labels in Cytoscape plotting
\end{ldescription}
\end{Arguments}
%
\begin{Value}
list of results elements 'annotation' = clusters and their annotation, 'network' = network of clusters
\end{Value}
%
\begin{Examples}
\begin{ExampleCode}
funjacc(gene_list, data_types='all', species='hsapiens', inflation=2)
\end{ExampleCode}
\end{Examples}
\HeaderA{funjacc\_res}{Data file of output from run\_funjacc function for testing create\_cytoscape\_network function}{funjacc.Rul.res}
\keyword{datasets}{funjacc\_res}
%
\begin{Description}
Data file of output from run\_funjacc function for testing create\_cytoscape\_network function
\end{Description}
%
\begin{Usage}
\begin{verbatim}
funjacc_res
\end{verbatim}
\end{Usage}
%
\begin{Format}
An object of class \code{list} of length 2.
\end{Format}
%
\begin{Author}
Adam Reid \email{ajr236@cam.ac.uk}
\end{Author}
\HeaderA{gene\_list}{Data file of gene lists for testing}{gene.Rul.list}
\keyword{datasets}{gene\_list}
%
\begin{Description}
Data file of gene lists for testing
\end{Description}
%
\begin{Usage}
\begin{verbatim}
gene_list
\end{verbatim}
\end{Usage}
%
\begin{Format}
An object of class \code{character} of length 79.
\end{Format}
%
\begin{Author}
Adam Reid \email{ajr236@cam.ac.uk}
\end{Author}
\HeaderA{run\_gprofiler}{Run gProfiler}{run.Rul.gprofiler}
%
\begin{Description}
Get gProfiler results for further processing
\end{Description}
%
\begin{Usage}
\begin{verbatim}
run_gprofiler(gene_list, organism)
\end{verbatim}
\end{Usage}
%
\begin{Arguments}
\begin{ldescription}
\item[\code{gene\_list}] List of genes in which to look for enriched terms

\item[\code{organism}] Name of species relating to gene list
\end{ldescription}
\end{Arguments}
\printindex{}
\end{document}
